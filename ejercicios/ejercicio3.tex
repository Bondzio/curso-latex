\documentclass{article}
\usepackage[spanish]{babel}
\usepackage[utf8]{inputenc}

\title{Ejercicio 3}
\author{Daniel Rodríguez}
\date{\today}

\begin{document}
\maketitle

En \LaTeX, es posible hacer una lista enumerada:
\begin{enumerate}
\item Primer elemento
\item Segundo elemento
\item Tercer elemento
\end{enumerate}

Y también, puedo hacer una lista no enumerada:
\begin{itemize}
\item Primer elemento
\item Segundo elemento
\item Tercer elemento
\end{itemize}

Adicionalmente, puedo crear listas dentro de listas:
\begin{itemize}
\item Aquí puedo poner otra lista:
    \begin{itemize}
    \item Primer elemento
    \item Segundo elemento
    \item Tercer elemento
    \end{itemize}
\item Y continúo con la lista original
\item Puedo
    \begin{enumerate}
    \item seguir
        \begin{itemize}
        \item tanto
            \begin{enumerate}
            \item como quiera.
            \end{enumerate}
        \end{itemize}
    \end{enumerate}
\end{itemize}
\end{document}