\documentclass[11pt, twocolumn]{article}
\usepackage[spanish]{babel}
\usepackage[utf8]{inputenc}
\usepackage{graphicx}

\title{Artículo extremamente importante sobre absolutamente nada}
\author{D.F. Rodríguez-Patiño}
\date{}

\begin{document}
    \maketitle
    \begin{abstract}
        El ejercicio es el siguiente. Modifique el archivo de \LaTeX{} de forma que al final de la introducción se mencione el contenido del artículo, de la siguiente forma: ``En la sección 1 se presenta la introducción; en la sección 2, el modelo; en la sección 3, los resultados y, finalmente, en la sección 4, las conclusiones''. Recuerde que la idea es que use referencias cruzadas e hipervínculos. Además, cree una referencia cruzada para la ecuación 1 en la sección de resultados analíticos. Finalmente, en la sección de conclusiones, añada la siguiente frase: ``A partir de los resultados obtenidos en las secciones 3.1 y 3.2, se concluye tal cosa.''.
    \end{abstract}

    \section{Introducción}
    Ut pharetra mauris vitae nibh porttitor placerat. Donec commodo magna non tincidunt egestas. Duis justo purus, sagittis a lorem at, tincidunt lobortis felis. Sed nec enim in velit hendrerit accumsan. Donec vel massa ac quam sodales pulvinar vel non lorem. Phasellus semper porta ligula, quis mollis risus viverra sit amet. Donec tempor sagittis condimentum. Mauris imperdiet purus ornare sem elementum, a mollis justo tincidunt. Praesent sit amet feugiat felis. Suspendisse eros ligula, blandit sed dolor vitae, ultrices efficitur felis. Donec id augue non eros consequat bibendum. Duis non elementum eros. Morbi vestibulum ultricies ligula ut lobortis. Suspendisse accumsan cursus augue, ac varius est facilisis quis. Vivamus fermentum molestie porttitor. Integer dictum eget neque non consequat.

    \section{Modelo}
    Curabitur laoreet vestibulum ligula, eget tristique turpis imperdiet nec. Nulla id euismod mi, id dignissim ex. Pellentesque mi dui, pulvinar eget tincidunt a, fringilla vitae nibh. Proin ac nulla fringilla sapien fermentum ornare non non libero. Phasellus sit amet consectetur sapien. Pellentesque pellentesque consequat tempus. 

    \begin{equation}
    \resizebox{0.8\columnwidth}{!} {$F(x) = A_0 + \sum_{n=1}^{N} \left[A_n \cos\left(\frac{2\pi nx}{P}\right) + B_n \sin\left(\frac{2\pi nx}{P}\right)\right]$}
    \end{equation}
    
    Quisque lacinia tellus vel rutrum feugiat. Nullam tempor, libero non auctor facilisis, dolor mauris pellentesque ligula, nec tempor arcu felis eu leo. Nunc vehicula nibh et gravida venenatis. Cras nunc turpis, viverra malesuada dapibus eu, pellentesque vitae risus.

    \section{Resultados}
    
    \subsection{Analíticos}
    In egestas eget elit et euismod. Proin id vehicula sapien. Integer semper, arcu a tempus fermentum, tellus neque tristique purus, semper faucibus nunc velit vel mi. Nam ultrices mauris ipsum, ut vehicula velit scelerisque in. Nam pellentesque aliquam urna, vitae ullamcorper quam pellentesque et. Quisque congue gravida ante, at ultrices nunc malesuada ac. Quisque ac nulla turpis. 
    
    \subsection{Simulados}
    Lorem ipsum dolor sit amet, consectetur adipiscing elit. Nam pulvinar nibh non urna convallis, at fermentum sem molestie. Integer nibh ante, maximus non lacinia et, imperdiet id magna. Vestibulum cursus orci non massa ornare, ac pellentesque dui vestibulum. Etiam egestas mi non feugiat ultricies. Donec cursus elementum elit, a porttitor massa.

    \section{Conclusiones}
    Donec consequat ex sit amet nisl feugiat, ut rutrum lorem eleifend. Nullam tristique, justo ac rutrum tristique, odio diam condimentum magna, eu rhoncus leo ligula vel leo. Nunc faucibus non orci at maximus. Nunc iaculis lectus id sem luctus, ac porttitor tortor porta. Donec ornare blandit quam, at sodales ligula pellentesque nec. Integer massa enim, %
    semper sit amet justo ut, posuere pellentesque ante. Donec porttitor mi leo, eget convallis erat vestibulum at. In id interdum neque, sed ullamcorper dolor.
\end{document}