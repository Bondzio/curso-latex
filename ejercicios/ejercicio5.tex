\documentclass{article}
\usepackage[spanish]{babel}
\usepackage[utf8]{inputenc}
\usepackage{amsmath}

\title{Ecuaciones de Maxwell}
\author{Daniel Felipe Rodríguez Patiño}
% \affiliation{Universidad Nacional de Colombia}
\date{\today}

\begin{document}
    \maketitle
    Las ecuaciones de Maxwell reciben su nombre en
    honor a James Clerk Maxwell, uno de los científicos
    más brillantes de la historia. Las ecuaciones que
    reciben su nombre, son el pilar del electromagnetismo
    y se enuncian a continuación:

    \begin{align}
        \vec{\nabla} \cdot \vec{E} \quad &= \quad \frac{\rho}{\varepsilon_0} && \textrm{Ley de Gauss} \\
        \vec{\nabla} \cdot \vec{B} \quad &= \quad 0 && \textrm{Ley de Gauss para el magnetismo} \\
        \vec{\nabla} \times \vec{E} \quad &= \quad -\frac{\partial\vec{B}}{\partial t} && \textrm{Ley de Inducción de Faraday} \\
        \vec{\nabla} \times \vec{B} \quad &= \quad \mu_0 \left(\varepsilon_0 \frac{\partial\vec{E}}{\partial t} + \vec{J} \right) && \textrm{Ley de Ampere}
    \end{align}
\end{document}