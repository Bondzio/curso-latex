\documentclass{article}
\usepackage[spanish]{babel}
\usepackage[utf8]{inputenc}
\usepackage{minted}
\usepackage{amsmath}
\usepackage{geometry}
\geometry{verbose,tmargin=3cm,bmargin=3cm,lmargin=7cm,rmargin=9cm}

\newenvironment{exampletwoup}
  {\VerbatimEnvironment
   \begin{VerbatimOut}{example.out}}
  {\end{VerbatimOut}
   \setlength{\parindent}{0pt}
   \fbox{\begin{tabular}{l|l}
   \begin{minipage}{0.55\linewidth}
     \inputminted[fontsize=\small,resetmargins]{latex}{example.out}
   \end{minipage} &
   \begin{minipage}{0.35\linewidth}
     \input{example.out}
   \end{minipage}
\end{tabular}}}


\begin{document}

% \begin{minted}[frame=single]{latex}
% \documentclass{article}
% \usepackage[spanish]{babel}
% \usepackage[utf8]{inputenc}

% \begin{document}
% Ahora podemos escribir tildes y
% virgulillas normalmente: Educación, piraña.
% \end{document}
% \end{minted}

% \vskip 2ex

% \begin{exampletwoup}
\begin{center}
\begin{minted}[frame=single]{latex}
\begin{align*}
(x+1)^3 &= (x+1)(x+1)(x+1) \\
        &= (x+1)(x^2+2x+1) \\
        &= x^3+3x^2+3x+1
\end{align*}
\end{minted}
\end{center}
% \end{exampletwoup}    

% \vskip 2ex

% \begin{exampletwoup}
% $\cos^2{\theta} + \sin^2{\theta} = 1$ \\

% $A_{\circ}=\pi r^2$
% \end{exampletwoup}    

\end{document}