\documentclass{beamer}
\usepackage[utf8]{inputenc}
\usepackage[spanish]{babel}
\usepackage{minted}

\usetheme{Madrid}
\usecolortheme{beaver}
\newcommand{\bftt}[1]{\textbf{\texttt{#1}}}
\newcommand{\cmd}[1]{{\color[HTML]{008000}\bftt{#1}}}
\newcommand{\bs}{\char`\\}
\newcommand{\cmdbs}[1]{\cmd{\bs#1}}

\title[Presentaciones en \LaTeX]
{\huge{Presentaciones \texttt{beamer} en \LaTeX}}
\subtitle{\textsl{Una introducción}}
\author{Daniel Felipe Rodríguez Patiño}
\institute[]{Facultad de Ciencias Exactas y Naturales \\
    Universidad Nacional de Colombia}
\date{\today}
% \logo{\includegraphics[height=1.2cm]{logo_unal}}

\begin{document}
    %%%%%%%%%%%%%%%%%%%%%%%%
    \frame{\titlepage}
    %%%%%%%%%%%%%%%%%%%%%%%%

    \section{Introducción}
    \begin{frame}[fragile]{\insertsection}
        Una presentación muy sencilla puede tomar la forma
        \begin{center}
            \begin{minipage}{0.6\linewidth}
                \inputminted[fontsize=\scriptsize, frame=single]{latex}{beamer_minimal.tex}
            \end{minipage}
        \end{center}
    \end{frame}

    %%%%%%%%%%%%%%%%%
    \begin{frame}[fragile]{\insertsection}
        \begin{itemize}
            \item \cmdbs{documentclass}article
        \end{itemize}
    \end{frame}
\end{document}